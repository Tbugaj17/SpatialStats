\documentclass{article}
\usepackage{amsmath}

\begin{document}

\title{Simulation of Poisson Processes}

\section*{Homogeneous Poisson Process Simulation on $[0, r]$}

In the homogeneous case, the intensity function \( \rho > 0 \) is constant.

\begin{enumerate}
    \item For \( i = 1, 2, \dots \) until \( x_i > r \), do the following:
    \begin{enumerate}
        \item Generate inter-arrival time \( t_i \sim \text{Exp}(\rho) \) using the inverse transform sampling:
        \[
        t_i = -\frac{\ln(u_i)}{\rho}, \quad u_i \sim \text{Unif}(0, 1).
        \]
        \item Let \( x_i = \sum_{j=1}^i t_j \) and set \( n = i - 1 \).
    \end{enumerate}
    \item Output the set of event locations \( X = \{x_1, x_2, \dots, x_n\} \).
\end{enumerate}

\section*{Inhomogeneous Poisson Process Simulation on $[0, r]$}

In the inhomogeneous case, the intensity function \( \rho(x) > 0 \) varies with position \( x \).

\begin{enumerate}
    \item For \( i = 1, 2, \dots \) until \( x_i > r \), do the following:
    \begin{enumerate}
        \item Generate inter-arrival time \( t_i \sim \text{Exp}(1) \) using the inverse transform sampling:
        \[
        t_i = -\ln(u_i), \quad u_i \sim \text{Unif}(0, 1).
        \]
        \item Let the cumulative intensity function be:
        \[
        H(x) = \int_0^x \rho(u) \, du.
        \]
        The event location \( x_i \) is then given by the inverse:
        \[
        x_i = H^{-1} \left( \sum_{j=1}^i t_j \right).
        \]
        \item Set \( n = i - 1 \).
    \end{enumerate}
    \item Output the set of event locations \( X = \{x_1, x_2, \dots, x_n\} \).
\end{enumerate}

In the homogeneous case, the process has constant rate $\rho$, and the event times are simply the cumulative sums of exponentially distributed inter-arrival times with rate $\rho$.

In the inhomogeneous case, the event times are still generated from exponential distributions, but the event locations are mapped through an inverse function $H^{-1}$, which is determined by the cumulative intensity function $H(x)$. 
The intensity function $\rho(x)$ varies with $x$, and this variation is accounted for by the mapping.
\end{document}
