\documentclass{article}
\usepackage{amsmath, amssymb}

\begin{document}

\textbf{Theorem (Existence of the Poisson Process):}  
The Poisson process $\text{Po}(S, \rho)$ exists.

\bigskip

\textbf{Corollary (Uniqueness of the Poisson Distribution):}  
The distribution of a Poisson process $\text{Po}(S, \rho)$ is completely determined by the following properties:
1. For any bounded Borel set $B \subset S$, the number of points in $B$ follows a Poisson distribution with mean $\int_B \rho(s) ds$.
2. The numbers of points in disjoint sets are independent.

\bigskip

\textbf{Theorem (Restriction of a Poisson Process):}  
Let $X \sim \text{Po}(S, \rho)$. The restriction of $X$ to a subset $B \subset S$, denoted $X \cap B$, is again a Poisson process with intensity $\rho$ restricted to $B$.

\bigskip

\textbf{Theorem (Superposition of Poisson Processes):}  
The union of independent Poisson processes in countably many disjoint domains is again a Poisson process.

\bigskip

\textbf{Theorem (Independent Scattering Property):}  
If $B_1, B_2, \dots$ are disjoint subsets of $S$, then the restricted processes $X_{B_1}, X_{B_2}, \dots$ are independent Poisson processes.

\bigskip

\textbf{Theorem (Characterisation of the Poisson Process):}  
The Poisson process is the only independently scattered point process.

\end{document}
