\documentclass{article}
\usepackage{amsmath, amssymb}

\begin{document}

\section*{Properties of a \(\sigma\)-Algebra}

Given a \(\sigma\)-algebra \(\mathcal{F}\) on a set \(S\), the following properties hold:

\subsection*{Contains the Empty Set}
Since \(S \in \mathcal{F}\) and \(\mathcal{F}\) is closed under complementation, the empty set must be in \(\mathcal{F}\):
\[
\emptyset = S \setminus S \in \mathcal{F}.
\]

\subsection*{Closed Under Finite Intersections}
If \(A, B \in \mathcal{F}\), then their intersection is also in \(\mathcal{F}\):
\[
A \cap B = S \setminus \left( (S \setminus A) \cup (S \setminus B) \right) \in \mathcal{F}.
\]
This follows from closure under complements and countable unions (applying it to two sets).

\subsection*{Closed Under Countable Intersections}
Since \(\mathcal{F}\) is closed under countable unions and complements, it follows from De Morgan’s laws that it is also closed under countable intersections:
\[
\bigcap_{n=1}^{\infty} A_n = S \setminus \left( \bigcup_{n=1}^{\infty} (S \setminus A_n) \right) \in \mathcal{F}.
\]

\subsection*{Monotonicity}
If \(A \in \mathcal{F}\) and \(B \subseteq A\), it does not necessarily imply that \(B \in \mathcal{F}\), unless \(\mathcal{F}\) contains all subsets of \(A\). This is why \(\mathcal{F}\) can be smaller than the power set \(2^S\).

\subsection*{Minimal and Maximal \(\sigma\)-Algebra}
The smallest \(\sigma\)-algebra on \(S\) is \(\{\emptyset, S\}\) (the trivial \(\sigma\)-algebra).  
The largest \(\sigma\)-algebra is the power set \(2^S\), containing all subsets of \(S\).

\subsection*{Why this matters:}
A \(\sigma\)-Algebra is essential for defining measurable functions and probability measures. It ensures that we can assign probabilities and integrals in a consistent way, avoiding paradoxes like non-measurable sets in probability theory.

\section*{The Borel \(\sigma\)-Algebra}

The Borel \(\sigma\)-algebra is a specific \(\sigma\)-algebra that plays a crucial role in measure theory and probability. It is the smallest \(\sigma\)-algebra containing all open sets in a given topological space, typically \(\mathbb{R}^n\).

\subsection*{Definition of the Borel \(\sigma\)-Algebra}
The Borel \(\sigma\)-algebra on \(\mathbb{R}^n\), denoted as \(\mathcal{B}(\mathbb{R}^n)\), is the \(\sigma\)-algebra generated by the open sets in \(\mathbb{R}^n\). 

This means \(\mathcal{B}(\mathbb{R}^n)\) is the smallest \(\sigma\)-algebra containing all open sets of \(\mathbb{R}^n\). 

Since a \(\sigma\)-algebra is closed under complements and countable operations, \(\mathcal{B}(\mathbb{R}^n)\) also contains closed sets, countable intersections and unions of open and closed sets, and more complex measurable sets.

\section*{Relation Between \(\sigma\)-Algebra and Borel Sets}

\subsection*{Borel \(\sigma\)-Algebra is a \(\sigma\)-Algebra}
The Borel \(\sigma\)-algebra satisfies all properties of a \(\sigma\)-algebra: it contains \(\mathbb{R}^n\), is closed under complements, and closed under countable unions.

However, not every \(\sigma\)-algebra is a Borel \(\sigma\)-algebra. A \(\sigma\)-algebra can be larger (e.g., the Lebesgue \(\sigma\)-algebra, which includes non-Borel sets).

\subsection*{Borel Sets are Elements of the Borel \(\sigma\)-Algebra}
A Borel set is any set that belongs to \(\mathcal{B}(\mathbb{R}^n)\). 

Examples: open sets, closed sets, countable unions of closed sets, countable intersections of open sets, and more.

\subsection*{Probability and Borel \(\sigma\)-Algebra}
In probability, we often define a probability space on \(\mathbb{R}^n\) as \(
(\mathbb{R}^n, \mathcal{B}(\mathbb{R}^n), P)
\), meaning that probability is assigned to Borel sets.

If a random variable \(X\) is measurable with respect to \(\mathcal{B}(\mathbb{R}^n)\), it means \( P(X \in B) \) is well-defined for all Borel sets \(B\).

\section*{Key Insight}
The Borel \(\sigma\)-algebra is a fundamental \(\sigma\)-algebra for defining measurable sets in topology and probability. It ensures that common sets (open, closed, countable unions/intersections) are measurable, allowing probability measures to be defined rigorously. However, it does not include all possible measurable sets (e.g., some Lebesgue-measurable sets are not Borel).

\end{document}
