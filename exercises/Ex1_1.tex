\documentclass{article}
\usepackage{amsmath} % For advanced math formatting

\title{Spatial Statistics Exercises Lecture 1}
\author{Tanja Bugajski}
\date{\today}

\begin{document}

\maketitle

\section*{Exercise 2}
Consider a binomial process on a Borel set $S\subseteq \mathbb{R}^d$ with $n$ points, where
each point follows a pdf $f$. \\
\(a\) Show that the intensity measure is given by
\[
\mu(B)=n\int_B f(x) \text{d}x
\]

\(b\)  Show that the void probability is given by
\[
v(B)=\left( 1-\int_B f(x) \text{d}x \right)^n
\]
for any bounded Borel set $S\subseteq \mathbb{R}^d$.
\section*{Exercise 3}
Why is it impossible for the binomial process to be stationary \textit{hint: think
about if the distribution of a point in the process then would be welldefined?}
\section*{4. Which of the following examples lead to the binomial process being isotropic:}
\begin{enumerate}
    \item[(a)] $f(x) = \frac{1}{|B(0, 1)|}$, where $S = B(0, 1)$ is a unit ball in $\mathbb{R}^d$ with the center at the origin and $|B(0, 1)| = \int_{B(0,1)} du$.
    
    \item[(b)] $f(x) = a^{-d}$ and $S$ is a $d$-dimensional box centered at the origin with side lengths $a > 0$.
    
    \item[(c)] $f(x)$ is the density of a $d$-dimensional normal distribution with mean vector 0 and covariance matrix $\sigma^2 I$, where $I$ is the identity matrix and $\sigma > 0$, and where $S = \mathbb{R}^d$.
\end{enumerate}

\section*{5. A small exercise in measures:}
\begin{enumerate}
    \item[(a)] Show that the intensity measure given by
    \[
    \mu(B) = \int_B \rho(x) \, dx
    \]
    for a non-negative function $\rho$ and $B \in \mathcal{B}$ is indeed a measure (more precisely, $\rho$ needs to be a so-called Borel function, which means that $\rho^{-1}(I)$ is a Borel set for any bounded interval $I$, but we don’t care about such measure theoretical details in this course).
    
    \item[(b)] On the other hand, argue why the void probability $v(B)$ for $B \in \mathcal{B}_0$ is not a measure on $S$. For instance, give a counter-example (hint: consider a binomial process).
\end{enumerate}

\section*{6. A creative exercise:}
\begin{enumerate}
    \item[(a)] Come up with your own example of a data type which is a point pattern.
    
    \item[(b)] Would you expect it to be clustered, regular, neither, or maybe some combination of the two?
    
    \item[(c)] Would you expect it to be homogeneous or inhomogeneous?
    \item[] 
    \item[(d)] Can you come up with some marks that could be associated with the point pattern?
    
    \item[(e)] Are there any covariates which might influence the point pattern?
\end{enumerate}





\end{document}
